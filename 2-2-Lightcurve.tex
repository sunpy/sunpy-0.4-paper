\subsection{Lightcurve}

Time series data and analyses are a fundamental aspect of solar physics for which many data sources are available. In recognition of this fact, SunPy provides a \textit{Lightcurve} object designed to provide a convenient and consistent interface for handling solar time series data. The main engine behind the \textit{Lightcurve} object is the \textit{pandas} data analysis library. This library contains a large amount of functionality for manipulating and analysing time series data, making it an ideal basis for \textit{Lightcurve}.  Like all other SunPy objects, \textit{Lightcurve} is a wrapper around a data object, which in this case is the \textit{pandas} object. 

Currently, the \textit{Lightcurve} object is compatible with the following data sources; the GOES X-ray Sensor (XRS), the SDO EUV Variability Experiment (EVE), and PROBA2/LYRA. For each of these instruments, a sub-class of the \textit{Lightcurve} object is initialized (e.g. \textit{GOESLightCurve, LYRALightCurve}) which inherits from \textit{Lightcurve} but allows instrument-specific functionality to be included. Future developments will introduce support for additional instruments and data products. 
Since time series data is generally relatively small and there is no established standard as to how it should be stored and distributed, each SunPy \textit{Lightcurve} object provides the ability to download its own data in its constructor.

A \textit{Lightcurve} object may be created using a range of different methods. Firstly, a \textit{Lightcurve} may be created for a specific instrument based on an input time range (see example below). In this example, the Lightcurve constructor searches remote sites for the GOES X-ray data specified by the time interval, downloads the required files and subsequently creates the object.

\begin{lstlisting}[language=Python]
from sunpy import lightcurve
goes=lightcurve.GOESLightCurve.create(`2011-06-07 06:00',
`2011-06-07 08:00')
goes.peek()

\end{lstlisting}

Alternatively, if the data file already exists on the local system, the \textit{Lightcurve} object may instead be initialized using that file as input (see example below).

\begin{lstlisting}[language=Python]
from sunpy import lightcurve
file=`lyra_20110607-000000_lev2_std.fits'
lyra=lightcurve.LYRALightCurve.create(file)
lyra.peek()

\end{lstlisting}

