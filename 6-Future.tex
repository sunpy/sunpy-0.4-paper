\section{Future of SunPy}\label{sec:future}

SunPy as a project has existed for roughly three years. In this time 
the code base has grown to over 17,000 lines. SunPy is already a 
useful package for the analysis of calibrated solar data, however, the code 
base is still evolving from release to release.
% RJH:  Again, lines of code is a weird measure.  Also, no way there is 110,000 lines that we
% actually wrote.  does this include docs, comments, and sphinx crap?
%SJM: I think this number is just wrong and never got changed.
%JI: needs new number.  Should be the number of lines in the 0.4 sunpy release???
%ays: I changed the number to ohloh's count unless there's a reason to think it's >110k
%schriste - using the ohloh number is consistent with sloccount so at least we are
%internally consistent +1
The primary focus of the 
%ays: Not actually discussed in Section \ref{sec:Intro}, so I removed the reference
SunPy library is the analysis and visualisation of `high-level' solar 
data. This means data that has been put through instrument processing 
and 
calibration routines, and contains full (WCS) coordinate information. 
The plan for SunPy is to continue development within this 
scope. The 
primary components of this plan are to provide a set of data types 
that are 
interchangeable with one another: e.g., if you slice a 
\texttt{MapCube} 
along one spatial coordinate, a \texttt{LightCurve} of intensity along the 
time range of 
the \texttt{MapCube} should be returned. To achieve this goal, all the 
data 
types need to share a unified coordinate system architecture so that 
each data 
type is aware of what the physical type of its data is and how 
operations on 
that data should be performed. This will enable useful operations
such as the coordinate and solar-rotation-aware 
overplotting of HELIO (Section \ref{ssec:helio}) and HEK
results (Section \ref{ssec:hek}) onto maps (Section \ref{ssec:map}).

In concert with the work on the data types, further integration with 
the 
\texttt{astropy} package will enable SunPy to incorporate many new features
with little effort. Collaboration and joint development with the 
Astropy project \citep{theastropycollaboration2013} is ongoing.
%schriste - shortened this bit as it was a bit vague and gave the impression that we may be subsumed by astropy