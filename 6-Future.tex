\section{Future of SunPy}
SunPy as a project has existed for roughly three years. In this time the code 
base has grown to over 110,000 lines. SunPy in it's current form is a useful 
package for the analysis of calibrated solar data, however, the code base is 
still very changeable from release to release.

The near-term future of the SunPy library is likely to continue this changeable 
nature with re-factoring and restructuring planned, especially within the 
core data types. The main focus upon the data types defined inside sunpy is to 
improve the interoperability and to redesign them in a more consistent manner.

In the past 12 months the collaboration with the Astropy project 
\cite{theastropycollaboration2013} has grown. The future of SunPy 
will continue this collaboration as the Astropy package provides many core 
tools that are applicable to all types of astronomy. For example the 
\texttt{astropy.units} submodule provides a simple and efficient way of 
representing physical units in Python data types, there are plans to 
incorporate this widely in the SunPy code-base.