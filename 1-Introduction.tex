\section{Introduction}
Science, including solar physics and astrophysics, is driven by the analysis 
and summarizing of data. Modern advances in sensor technology combined with 
the availability of cheap storage and large bandwidth communications has led to 
a rapid and overwhelming increase in amount of data which scientists are responsible
for analyzing. For example, NASA's Solar 
Dynamics Observatory (SDO) satellite records over 1 TB of data per day. To analyze this 
mountain of data and enable scientific discovery requires increasingly complex software 
tools. Standardised, easy to use,
and transparent software tools are a key enabling technology in the scientific endeavor
so that the community can build upon a common foundation.
%schriste - the opening paragraph still needs work

SunPy aims to provide a free, open source and openly developed software package 
for the analysis and visualisation of solar data. SunPy makes use of the Python 
programming language and the breath of high-quality scientific focused packages 
available for it. The Python programming language is a general purpose, 
powerful and easy to learn high-level programming language. Python is one of 
the top ten most popular programming languages in the world according to the 
2014 TIOBE Index\footnote{\url{http://www.tiobe.com/index.php/content/paperinfo/tpci/index.html}},
and is widely used outside of scientific fields in web development, education 
and `big data' analytics.

The development of a package such as SunPy in Python is made possible by the 
rich scientific Python ecosystem of packages available. Core packages in this 
ecosystem such as \texttt{NumPy}, \texttt{SciPy} and \texttt{matplotlib} 
provide the basic functionality expected of a scientific programming language, 
\textit{i.e.} array manipulation, core numerical algorithms and visualisation. 
Build upon these foundations, packages such as \texttt{Astropy}, \texttt{pandas} and 
\texttt{scikit-image} provide more domain-specific functionality.

The design philosophy of SunPy is to provide a clean, simple to use, and well 
structured package that provides the \textit{core} tools for solar physics. The 
primary focus of SunPy's early development is to provide specialised, linked, 
datatypes that allow the acquisition, processing and visualisation of all types 
of solar data.

The purpose of this paper is to provide an overview of SunPy's current 
capabilities, an overview of the development model and community aspects of the 
SunPy project as well as future plans. The latest release of SunPy (0.4) from
the sunpy website (\url{http://sunpy.org}) or can be installed using 
the Python package index (\mbox{\url{http://pypi.python.org/pypi}}).
%schriste - the mbox prevents latex from breaking the url across different lines.