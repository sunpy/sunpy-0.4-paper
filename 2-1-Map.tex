\subsection{Map}
Map is a 2D spatial data type, primarily used for images of the Sun and 
inner heliosphere. It provides a wrapper around a data array (\texttt{numpy} 
\texttt{ndarray}) and gives easy access to standard meta data in the header of 
the image, via the \texttt{.meta} attribute as well as some convienience 
properties for more standard meta data, \textit{i.e.} \texttt{.instrument}, 
\texttt{.wavelength} or \texttt{.coordinate\_system}.
The \texttt{Map} data type provides convenience methods for many functions 
such as, rotation and re-sampling as well as convenience visualisation 
functions, providing a easy to use as well as powerful interface.

The design of the map submodule is such that each instrument or 
detector can subclass the parent \texttt{GenericMap} class to implement 
special meta data handling or other data specific functions. Each subclass 
of \texttt{GenericMap} can register with the \texttt{Map} factory class and 
by implementing a method that returns \texttt{True} if the meta data 
matches meta data for that instrument or detector, the \texttt{Map} factory 
will automatically return an instance of the specific \texttt{GenericMap} 
subclass. As of version 0.4 SunPy has \texttt{Map} specialisations for the 
following instruments: \textit{IRIS} SJI frames, \textit{SDO/AIA} and 
\textit{HMI}, \textit{SOHO/EIT} and	\textit{LASCO}, \textit{STEREO/EUVI} and 
\textit{COR}, \textit{YOHKOH/SXT}, \textit{RHESSI}, \textit{PROBA2/SWAP} and 
\textit{HINODE/XRT}.

As can been seen in Listing \ref{code:aia_1} the map datatype parses header 
information for many purposes including visualisation on correct physical axes 
and limb plotting as well as making this information easily available to the 
user.

\begin{listing}[H]
\begin{minted}{pycon}

>>> import sunpy.map

>>> aiamap = sunpy.map.Map('aia_file.fits')
>>> aiamap
SunPy AIAMap
---------
Observatory: SDO
Instrument:  AIA
Detector:    AIA
Measurement: 171
Obs Date:    2011-03-19T10:54:00.34
dt:          1.999601
Dimension:   [1024, 1024]
[dx, dy] =   [2.400000, 2.400000]

array([[ 0.3125, -0.0625, -0.125 , ...,  0.625 , -0.625 ,  0.    ],
       [ 1.    ,  0.1875, -0.8125, ...,  0.625 , -0.625 ,  0.    ],
       [-1.1875,  0.375 , -0.5   , ..., -0.125 , -0.625 , -1.1875],
       ..., 
       [-0.625 ,  0.0625, -0.3125, ...,  0.125 ,  0.125 ,  0.125 ],
       [ 0.5625,  0.0625,  0.5625, ..., -0.0625, -0.0625,  0.    ],
       [ 0.5   , -0.125 ,  0.4375, ...,  0.6875,  0.6875,  0.6875]])

>>> aia.peek()
\end{minted}
\includegraphics[width=0.8\columnwidth]{aia_map_example}
\caption{Demonstration of the \texttt{AIAMap} specilisation of 
\texttt{GenericMap}. The Map is created from a \textit{AIA} FITS file and the 
key meta data and array overview is printed. Then a quick view plot is created 
by using the \texttt{.peek()} method.}
\label{code:aia_1}
\end{listing}

As well as providing the base classes the map submodule provides two 
collection classes, \texttt{CompositeMap} and \texttt{MapCube}, for 
temporally and spatially aligned data respectively. \texttt{MapCube} 
provides methods for animation of its series of \texttt{Map} objects. 
\texttt{CompositeMap} provides methods for overlaying spatially aligned 
data, with support for visualisation of images and contour lines overlaid 
upon each other.

\begin{listing}[H]
\begin{minted}{python}
import matplotlib.pyplot as plt
import sunpy.map

#Create a composite map from two files
compmap = sunpy.map.Map('aia_1600_image.fits', 'RHESSI_image.fits', 
composite=True)

#Set the RHESSI image (index 1) to have contour levels and a red colour map.
compmap.set_levels(1,range(0,50,5),percent=True)
compmap.set_colors(1,'Reds_r')

#Plot the result and crop
ax = plt.subplot()
compmap.plot()
ax.axis([200,600,-600,-200])
\end{minted}
\includegraphics[width=0.8\columnwidth]{comp_map_example}
\caption{Example demonstrating a CompositeMap plot, using contours and how 
SunPy integrates with matplotlib's pyplot functional interface.}
\label{code:compmap_1}
\end{listing}

\begin{listing}[H]
\begin{minted}{python}
import sunpy.map

compmap = sunpy.map.Map('aia_lev1_171a_2014_01*fits', cube=True)
compmap.peek()
\end{minted}
\includegraphics[width=0.8\columnwidth]{aia_cube_controls}
\caption{An example showing creation of a MapCube from a glob file search. The 
resultant plot makes use of matplotlib's interactive widgets to allow scrolling 
through the MapCube.}
\label{code:mapcube_1}
\end{listing}

\begin{listing}[h]
\begin{minted}{python}
import sunpy.map
from matplotlib import animation

mapc = sunpy.map.Map('aia_lev1_171a_2014_01*fits', cube=True)
anim = mapc.plot()
Writer = animation.writers['ffmpeg']
writer = Writer(fps=10, metadata=dict(artist='SunPy'), bitrate=1800)
anim.save('aia_cube.ogv')
\end{minted}
\caption{Example showing how to save a video animation from a MapCube, using 
matplotlib's animation framework.}
\label{code:mapcube_2}
\end{listing}