\section{Core Data Types}
The primary focus of the SunPy package is to provide data structures 
that are specifically designed for the various types of solar physics data, namely
images, time series, and spectra. SunPy provides three data core
objects to support these data types, 2D spatial data 
(\texttt{Map}), 1D temporal series (\texttt{LightCurve}) and 1 and 2D spectra 
(\texttt{Spectrum} and \texttt{Spectrogram}).

The philosophy of these objects is that they should provide a access to the original data
along with associated meta data and provide appropriate convenience functions to
enable data analysis and visualization. For all core data types, the data is stored in 
stored in the \texttt{.data}) attribute while the meta data is stored 
in the \texttt{.meta} attribute. It is 
possible to instantiate the data types from various different 
sources, \textit{e.g.} files, urls, arrays or time ranges.  In order to provide
instrument-specific specialisation, all the core SunPy data types support specialisation, 
\textit{i.e.} \texttt{Map} has an \texttt{AIAMap} sub-type. 
Data visualisation is provided by two functions; \texttt{peek()}, allows quick view plotting
, while more fine-grained control plotting that integrates with 
\texttt{matplotlib} is provided by \texttt{plot()}.

This section will give a brief overview of the \textit{current} functionality 
of each of these modules.

\subsection{Map}\label{ssec:map}
The map data type stores 2D spatial data, such as images of the Sun and 
inner heliosphere. It provides: a wrapper around a \texttt{numpy} data array, 
the images associated spatial coordinates, and other metadata. The \texttt{Map} 
class provides methods for typical operations on 2D data, such as rotation and 
re-sampling, as well as visualisation functionality.
The \texttt{Map} class also provides a convenient interface for loading data 
from a variety of sources, including from FITS (Flexible Image Transport System) files, the standard format for storing image data in solar physics and astrophysics. An example of creating a \texttt{Map} object from a FITS file is shown in 
Listing~\ref{code:aia_1}.


The architecture of the map subpackage consists of a template map called
\texttt{GenericMap}, which is a subclass of \texttt{astropy.nddata.NDData}. 
\texttt{NDData} is a generic wrapper around a \texttt{numpy.ndarray} with a 
\texttt{meta} attribute to store metadata.
As \texttt{NDData} is currently still in development, \texttt{GenericMap} does 
not yet make full use of its capabilities, but this inheritance structure 
provides for future integration with \texttt{astropy}. In order to provide 
instrument- or detector-specific integration, \texttt{GenericMap} is designed
to be subclassed. Each subclass of \texttt{GenericMap} can register 
with the \texttt{Map} creation factory, which will then automatically return an instance
of the specific \texttt{GenericMap} subclass dependent upon the data provided. 
SunPy v0.4 has \texttt{GenericMap} specialisations for the following 
instruments: 
\textit{Yohkoh}/SXT, \textit{SOHO}/EIT and LASCO, \textit{RHESSI}, 
\textit{STEREO}/EUVI and COR, \textit{Hinode}/XRT,
\textit{PROBA2}/SWAP, \textit{SDO}/AIA and HMI, 
and \textit{IRIS} SJI (slit-jaw imager) frames. 
                        
The \texttt{GenericMap} class stores all of the metadata retrieved from the header of
the image file in the \texttt{meta} attribute and provides convenience 
properties for commonly accessed metadata: e.g., \texttt{instrument}, 
\texttt{wavelength} or \texttt{coordinate\_system}.
These properties are dynamic mappings to the underlying metadata and all methods 
of the \textit{GenericMap} class modify the meta data where needed.
For example, if \verb|aiamap.meta[`instrume']| is modified then \verb|aiamap.instrument| 
will reflect this change.
Currently this is implemented by not preserving the keywords of the input data,
instead modifying meta data to a set of ``standard" keys supported by SunPy.
Listing \ref{code:aia_1} demonstrates the quick-look functionality of 
\texttt{Map}.

\begin{listing}[H]
\pythoncode{pycode_map1.txt}
\begin{center}
\includegraphics[width=0.8\columnwidth]{aia_map_example}
\end{center}
\caption{Example of the \texttt{AIAMap} specialisation of 
\texttt{GenericMap}. The map is created from an \textit{SDO}/AIA FITS file, a cutout
of the full map is created by specifying the desired solar-$x$ and solar-$y$ ranges of the plot in data coordinates (in this case, arcseconds), and then a quick-view plot is created with lines of heliographic longitude and latitude over-plotted.}
\label{code:aia_1}
\end{listing}

In addition to the data-type classes, the \texttt{map} subpackage provides two 
collection classes, \texttt{CompositeMap} and \texttt{MapCube}, for 
spatially and temporally aligned data respectively.
\texttt{CompositeMap} provides methods for overlaying spatially aligned 
data, with support for visualisation of images and contour lines overlaid 
upon each other.
\texttt{MapCube} provides methods for animation of its series of \texttt{Map} 
objects. Listings~\ref{code:compmap_1} and \ref{code:mapcube_1} show how to 
interact with these classes.

\begin{listing}[H]
\pythoncode{pycode_map2.txt}
\begin{center}
\includegraphics[width=0.8\columnwidth]{comp_map_example}
\end{center}
\caption{Example showing the functionality of \texttt{CompositeMap}, with RHESSI X-ray image data composited
on top of an \textit{SDO}/AIA 1600 $\AA$ image. The \texttt{CompositeMap} is plotted using the integration with the \texttt{matplotlib.pyplot} interface.}
\label{code:compmap_1}
\end{listing}

\begin{listing}[H]
\pythoncode{pycode_map3.txt}
\begin{center}
\includegraphics[width=0.8\columnwidth]{aia_cube_controls}
\end{center}
\caption{Example showing the creation of a \texttt{MapCube} from a list of AIA image files. The 
resultant plot makes use of \texttt{matplotlib}'s interactive widgets to allow scrolling 
through the \texttt{MapCube}.}
\label{code:mapcube_1}
\end{listing}
\subsection{Lightcurve}\label{ssec:lightcurve}

Time series data and their analyses are a fundamental aspect of solar
physics for which many data sources are available.
SunPy provides a \texttt{LightCurve} class
with a convenient and consistent interface for handling solar time-series
data.  The main engine behind the \texttt{LightCurve} class is
the \href{http://pandas.pydata.org}{\texttt{pandas}} data-analysis library, and 
\texttt{LightCurve}'s \texttt{data} attribute is a \texttt{pandas.DataFrame} 
class.
The \texttt{pandas} library contains a large amount
of functionality for manipulating and analysing time-series data,
making it an ideal basis for \texttt{LightCurve}.  \texttt{LightCurve}
assumes that the input data are time-ordered list(s) of numbers, and each
list becomes a column in the \texttt{pandas} DataFrame class.

Currently, the \texttt{LightCurve} class is compatible with the
following data sources: the GOES X-ray Sensor (XRS), PROBA2/LYRA, and
the SDO EUV Variability Experiment (EVE; only the level ``OCS'' and
% Do we need to explain what OCS means?
% RJH: I'd argue that everything other than (EVE) can be cut, but I'm not 
% confident enough to make that cut.
average CSV files -- see \url{http://lasp.colorado.edu/home/eve/data/}
for more detail).  For each of these instruments, a subclass of the
\texttt{LightCurve} object is initialised
(\textit{e.g.}, \texttt{GOESLightCurve}) which inherits from
\texttt{LightCurve} but allows instrument-specific functionality to be
included.  Future developments will introduce support for additional
instruments and data products, as well as implementing a factory interface 
similar to that of \texttt{Map}.  Since there is no established standard
as to how time-series data should be stored and distributed, each SunPy 
\texttt{LightCurve} object sub-class provides the ability to download its corresponding 
specific data format in its constructor and parse that file type.

A \texttt{LightCurve} object may be created using a number of different methods. 
For example, a \texttt{LightCurve} may be created for a specific instrument given
an input time range. In Listing~\ref{code:goes_lc}, 
the \texttt{LightCurve} constructor searches a remote source for the GOES X-ray 
data specified by the time interval, downloads the required files and 
subsequently creates and plots the object.

\begin{listing}[H]
\begin{minted}[bgcolor=bg]{pycon}
>>> from sunpy import lightcurve
>>> from sunpy.time import TimeRange
>>> goes = lightcurve.GOESLightCurve.create('2011-06-07 06:00',
...                                         '2011-06-07 08:00')
>>> goes.peek()
%removed the resample part of the example - really we need to fix truncate!
\end{minted}
\begin{center}
\includegraphics[width=10cm]{goes_lightcurve.pdf}
\end{center}
\caption{Example retrieval of a GOES lightcurve for the time interval 
06:00--08:00 UT on 2011 June 7 using a time range, and the output of the 
\texttt{peek()} method.}
\label{code:goes_lc}
\end{listing}
%schriste - note to self, fix this example, add resampled points on top of plot?

Alternatively, if the data file already exists on the local system, the 
\texttt{Lightcurve} object may instead be initialised using that file as input.
Once the \texttt{Lightcurve} has been created, it may be manipulated in 
a variety of ways in order to perform time-series analysis.

%Stuart: Where did the pandas section go?
%ARI - looks like it's up top

\subsection{Spectra}\label{sec:spectra}
%schriste - this section needs major work

%ayshih - this whole paragraph can be deleted
Spectroscopy consists in the study of the radiative energy related to the wavelength.
An spectrum is normally obtained by observing a range of frequencies at the same time; 
it can be measured for example in radio wavelengths by a frequency sweep with 
a radio receiver or in visible and ultraviolet by the dispersion of the incident light 
through a diffraction grating or prism.
The analysis of the resultant spectrum can provide properties of the plasma observed 
such temperature, density, speed, etc.
Therefore, spectroscopy provides to the solar physicists invaluable information about 
the composition and the physical properties of the Sun.  

SunPy aims to provide broad support for solar spectroscopy instruments, however the 
variety and complexity of such datasets makes it very challenging to add support to them all 
in a general way.
The \texttt{spectra} module implements a \texttt{Spectrum} class for 1D data
(intensity versus frequency) and a \texttt{Spectrogram} class for 2D data
(intensity versus time and frequency.
Each of these classes use a \texttt{numpy.ndarray} class as its \texttt{.data} attribute.
These two classes were implemented by funding provided by the Astrophysics Research 
Group at Trinity College Dublin.

As of SunPy 0.4, the \texttt{Spectrogram} class supports radio spectra from the e-Callisto 
solar radio spectrometer network (\url{http://www.e-callisto.org/})
and STEREO/SWAVES spectrograms.
As with other SunPy data types, the \texttt{Spectrogram} class has been
built so each instrument initialises using a subclass containing the instrument-specific 
functionalities.
The common functionality provided by the base \texttt{Spectrogram} class includes
reading data,
joining different time ranges and frequencies,
performing frequency-dependent background subtraction,
and convenient visualization and sampling of the data.

Listing \ref{code:spectra} shows how the \texttt{CallistoSpectrogram} object retrieves
from the online data archive the files that are between the time range specified and
the observatory of interest.  In this case the data is requested from \textit{BIR},
which is the code name of the
Rosse Observatory at Birr Castle in Ireland (\url{http://www.rosseobservatory.ie}).
When the data is requested using the \texttt{.from\_range} function, the object merges
into a single spectrogram all the files downloaded, each with 15 minutes observation data 
and in some cases with different files for different frequency ranges.  
In the example shown here BIR observed in two frequency ranges: 20--90\,MHz and 55--355\,MHz.
Since the data is not evenly spaced in the frequency range, the \texttt{Spectrogram} object
linearise the frequency axis for a better analisys, as shown in the resultant figures.
It is also shown in the example the effect of the implemented background substraction and
\texttt{.peek} method.

\begin{listing}[H]
\begin{minted}[bgcolor=bg]{pycon}
>>> from SunPy.spectra.sources.callisto import CallistoSpectrogram
>>> tstart, tend = "2011-06-07T06:00:00", "2011-06-07T07:45:00"
>>> callisto = CallistoSpectrogram.from_range("BIR", tstart, tend)
>>> callisto_nobg = callisto.substract_bg()
>>> callisto_nobg.peek(vmin = 0)
\end{minted}
\begin{center}
\includegraphics[width=0.8\columnwidth]{callisto_nobg}
\end{center}
\caption{Example of how \texttt{CallistoSpectrogram} retrieves the data
for the time range and observatory requested, merges it all and removes the background
signal.}
\label{code:spectra}
\end{listing}

% Download Callisto
% Merge multiple time-ranges / frequencies (just work from downlad!)
% Merge callisto with swaves



\subsection{Visualisation}
\label{subsec:Viz}
As has been demonstrated in this section, the core SunPy datatypes 
include visualisation code that is specialised to that data type. 
These visualisation methods all currently utilise the matplotlib 
package, and are designed in such a way that they integrate well with 
the pyplot functional interface of matplotlib, as demonstrated in 
Figure \ref{code:mpl_1}.

%schriste - I vote for removing all of the following code as being too down in the weeds
\begin{listing}[H]
\begin{minipage}{0.49\columnwidth}
\begin{minted}[bgcolor=bg]{python}
import sunpy.map
import matplotlib.pyplot as plt

#Read in a aia file using SunPy
aiamap = SunPy.map.Map('aia.fits')

#Create the figure and axis 
#explicitly
fig = plt.figure()
ax = plt.subplot()

#Create a image plot with raw data
im = ax.imshow(aiamap.data,...)
#add a colour bar
plt.colorbar(im,ax=ax)

#Add a title
plt.title("My AIA plot")
plt.show()
\end{minted}
\end{minipage}
\begin{minipage}{0.49\columnwidth}
\begin{minted}[bgcolor=bg]{python}
import SunPy.map
import matplotlib.pyplot as plt

#Read in a aia file using SunPy
aiamap = SunPy.map.Map('aia.fits')

#Create the figure and axis 
#explicitly
fig = plt.figure()
ax = plt.subplot()

#Plot the AIA map
im = aiamap.plot(axes=ax)
#add a colour bar
plt.colorbar(im,ax=ax)

#Add a title
plt.title("My AIA plot")
plt.show()
\end{minted}
\end{minipage}
\caption{A simple example of how SunPy's plotting functions provide similar 
behaviour to matplotlib's pyplot interface.}
\label{code:mpl_1}
\end{listing}

This design philosophy makes the behaviour of SunPy's visualisation 
routines intuitive to those who already understand how matplotlib's 
interface is designed as well as allowing the use of the standard 
matplotlib commands to change the plot, \textit{e.g.} set the title. 
More technically what is done in the plotting routines is firstly, to 
make sure that the current state of the figure and axes is mantained 
in the same way as it is inside the \texttt{pyplot} functions. For 
\texttt{Map} this means setting \texttt{pyplot.sci()} or set current 
image, so that \texttt{plt.colorbar()} 
automatically detects the plotted \texttt{AxesImage} object. 
Secondly, it also means returning the same type of object as the 
\texttt{pyplot} interfaces function, \textit{i.e.} for \texttt{Map} 
it returns the \texttt{AxesImage} instance which is the same type of 
return as \texttt{pyplot.imshow}.
